\chapter{Bevezetés}
\label{ch:intro}

A hulladékszennyezés komoly problémát jelent a természet számára \cite{kibria2023PlasticWaste}. Emiatt számos szervezet mozdul abba az irányba, hogy tisztábbá tegye a bolygónkat. Egy ilyen szervezet a PET Kupa, akik elsősorban folyómenti hulladékgyűjtéssel foglalkoznak elsősorban Magyarországon, de figyelmük kiterjed a szomszédos országokra is. Az egyik nagy kihívás a szemétgyűjtésben ezeknek a területeknek megtalálása. Sok emberi és pénzügyi erőforrást igényel a hulladéklerakók megtalálása a folyók mentén, hiszen jármüvekkel valakinek végig kell haladnia egy hosszabb területen, csak azért, hogy felmérje, hogy hol van hulladék. Ehhez jelentős mennyiségű üzemanyagot kell elhasználni \todo{Ide esetleg egy hivatkozás?}. Emiatt hatékonyabb eszközökre van szükségünk, hogy ezt a folyamatot felgyorsítsuk. Ennek fényében a PET Kupa felkereste az egyetemünket azzal a kéréssel, hogy olyan eszközöket fejlesszünk le, melyek automatikusan képesek lesznek hulladékot detektálni a folyók mentén.

A dolgozatomban bemutatok egy Random Forest modell-t, mely a kutatólaborban már lefejlesztett modellre épül \cite{magyar2023}. A bemutatott modell javít a korábbi modell problémáin, illetve nagyobb megbízhatósággal találja meg a hulladékot a folyókon és a folyók mentén. A modell eredményei integrálásra kerülnek a Tiszta Tisza webalkalmazásba, ahol több napon keresztül történő detektálás eredménye lesz összesítve és megjelenítve a felhasználók számára \todo{Hivatkozni a weboldalra}.