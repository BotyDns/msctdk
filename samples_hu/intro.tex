\chapter{Bevezetés}
\label{ch:intro}

\todo{szinkronizálni az absztrakttal}A hulladékszennyezés komoly problémát jelent a természet számára \cite{kibria2023PlasticWaste}. Emiatt számos szervezet mozdul abba az irányba, hogy tisztábbá tegye a bolygónkat. Egy ilyen szervezet a PET Kupa, akik folyómenti hulladékgyűjtéssel foglalkoznak elsősorban Magyarországon, de figyelmük kiterjed a szomszédos országokra is. Az egyik nagy kihívás a szemétgyűjtésben a hulladékkal szennyezett területeknek a hatékony megtalálása. Sok erőforrást igényel a hulladéklerakók megtalálása a folyók mentén, hiszen sokszor járművel kell valakinek végig haladnia egy hosszabb területen, azért, hogy felmérje, hogy hol van hulladék. A folyók árterén elhelyezett hulladékok még nagyobb problémát jelentenek, hiszen dagály idejében a hulladékot elmossa a víz és ez a folyó további szakaszaira lesz szétszórva miközben nagy károkat okoz a folyó élővilágának, illetve szennyezi a folyóvizet \cite{nyberg2023, vanEmmerik2023}. Emiatt szükségünk van olyan eszközökre, melyekkel hamar lehet detektálni a szennyezett területeket, hogy ezeket minél hatékonyabban meglehessen tisztítani. Ennek fényében a PET Kupa felkereste az egyetemünket azzal a kéréssel, hogy olyan eszközöket fejlesszünk le, melyek automatikusan képesek lesznek hulladékot detektálni a folyók mentén.

A dolgozatomban bemutatok egy Random Forest modellt \cite{breiman2001}, mely a kutatólaborban már lefejlesztett modellre épül \cite{magyar2023}. A bemutatott modell javít a korábbi megoldás pontosságán, illetve nagyobb megbízhatósággal találja meg a hulladékot a folyókon és a folyók mentén. A modell eredményei integrálásra kerülnek a Tiszta Tisza webalkalmazásba, ahol több napon keresztül történő detektálás eredménye lesz összesítve és megjelenítve a felhasználók számára \todo{Hivatkozni a weboldalra}. Ezen felül a dolgozatban tárgyalni fogok más kutatást is, mely a hulladékdetektálás problémájával foglalkozik. Ezen kívül kiegészítem a kutatólabor meglevő szoftveres eszközeit annak érdekében, hogy a laborban zajló munka gördülékenyebb legyen.

A kutatás hozzáadott terméke egy olyan adathalmaz, mely alkalmas más hulladékdetektálási modellek betanítására is. Az adathalmaz elsősorban szárazföldi romániai hulladéklerakókról készített PlanetScope műholdfelvételeket tartalmaz, melyek kézzel voltak annotálva. Az adatok georeferálva vannak, így ezeket könnyen meg lehet vizsgálni, illetve ki lehet egészíteni.

Továbbá bemutatom, hogy milyen módszerekkel próbáltam tovább javítani a modell eredményein. Ilyen módszer például a főkomponens analízis, a képnormalizálás, illetve az évszakokra bontás.

\section{A kutatólabor eddigi eredményei}

A térinformatikai kutatólaborban már fejlesztésre került egy szerveralkalmazás, mely minden nap a Planet szervereiről letölti a legfrissebb felvételeket a vizsgált területekről, és lefuttatja ezeken a képeken az akkori modellt. Ezen felül készült egy webalkalmazás is, ami erről a szerverről letölti az eredményeket, és megjeleníti ezeket, összehasonlításra. A kutatólabor rendelkezik egy asztali alkalmazással is, mellyel hatékonyan elő lehet állítani tanítóadatokat. A kutatásom elősegítéséhez ezeket az alkalmazásokat használtam, illetve bővítettem a \ref{ch:application-improvement} fejezetben leírtak szerint.