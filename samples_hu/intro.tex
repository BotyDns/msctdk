\chapter{Bevezetés}
\label{ch:intro}

Napjainkban nagy kihívást jelentenek a hulladékkal szennyezett területek: ezek kárt okoznak a környező élővilágban, illetve emberek számára is egészségügyi kockázatot jelentenek. \cite{kibria2023PlasticWaste}. Ezért kialakultak olyan önkéntes szervezetek, mint a PET Kupa, akik hulladékgyűjtéssel foglalkoznak elsősorban a magyarországi folyók mentén, de szomszédos országokra is kiterjed a figyelmük. Az egyik nagy kihívás a szemétgyűjtésben a hulladékkal szennyezett területeknek a hatékony megtalálása. Sok erőforrást igényel a szemétlerakók megtalálása a folyók mentén, hiszen sokszor járművel kell valakinek végig haladnia egy hosszabb területen, azért, hogy felmérje, hogy hol van hulladék. A folyók árterén elhelyezett hulladék még nagyobb problémát jelent, hiszen áradás idejében a szemetet elmossa a víz és ez a folyó további szakaszaira lesz szétszórva miközben nagy károkat okoz a folyó élővilágának, illetve szennyezi a folyóvizet \cite{nyberg2023, vanEmmerik2023}. Emiatt szükségünk van olyan eszközökre, melyekkel hamar lehet detektálni a szennyezett területeket, hogy ezeket minél hatékonyabban meglehessen tisztítani. Az ELTE IK Térinformatikai Labor és a PET Kupa együttműködésében olyan eszközöket fejlesztünk, melyek automatikusan képesek lesznek hulladékot detektálni a folyók mentén.

A dolgozatomban bemutatok egy Random Forest modellt \cite{breiman2001}, mely a kutatólaborban már korábban kifejlesztett módszerekre épül \cite{magyar2023}. A bemutatott modell javít a korábbi megoldás \textit{false positive} arányain, miközben tovább is képes hulladékot detektálni. A modell eredményei integrálásra kerülnek a Tiszta Tisza webalkalmazásba \cite{tisztatisza2024}\footnote{A Tiszta-Tisza webalkalmazás hivatalos weboldala a hivatkozott címen található, de a dolgozatomban említett fejlesztések jelenleg a https://gis.inf.elte.hu/tiszta-tisza/ oldalon érhetőek el.}, ahol több napon keresztül történő detektálás eredménye lesz összesítve és megjelenítve a felhasználók számára.
\todo[inline,color=blue!50]{Hivatkozd meg a https://tisztatiszaterkep.hu/ oldalt, de említsd meg, hogy a prototípus egyelőre a https://gis.inf.elte.hu/tiszta-tisza/ oldalon érhető el. Ez mehet akár csak lábjegyzetbe. Botond: Rendben, lábjegyzetben megemlítettem.}

A kutatás hozzáadott terméke egy olyan adathalmaz, mely alkalmas más hulladékdetektálási modellek betanítására is. Az adathalmaz elsősorban szárazföldi romániai hulladéklerakókról készített PlanetScope műholdfelvételeket tartalmaz, melyek kézzel voltak annotálva. Az adatok georeferálva vannak, így ezeket könnyen meg lehet vizsgálni, illetve ki lehet egészíteni. A tanítóadatok hatékony feldolgozása érdekében bemutatok egy módosítást a meglevő index-számoló algoritmusban, mellyel lényegesen csökkentem a felvételek feldolgozási idejét.

Kutatásomban a modell teljesítményének javítását további módszerekkel, így például főkomponens analízissel, képnormalizálással, vízmaszkolással, illetve az annotált felvételek évszakokra bontásával specifikus modellek készítésével is vizsgálom.

\section{A kutatólabor eddigi eredményei}

A ELTE IK Térinformatikai Kutatólaborában már betanításra került egy Random Forest modell, mely folyómenti hulladék detektálására alkalmas. Az én dolgozatomban, a kutatólabor meglevő tudására építve, továbbfejlesztem ezt a modellt, hogy kevesebb \textit{false positive}-al találja meg a szeméttel szennyezett területeket. Továbbra implementálásra került egy szerveralkalmazás, mely minden nap a Planet szervereiről letölti a legfrissebb felvételeket a vizsgált területekről, és lefuttatja ezeken a képeken az akkori modellt. Ezen felül készült egy webalkalmazás is, ami erről a szerverről letölti az eredményeket, és megjeleníti ezeket, összehasonlításra. A kutatólabor rendelkezik egy asztali alkalmazással is, mellyel hatékonyan elő lehet állítani tanító adatokat. A kutatásom elősegítéséhez ezeket az alkalmazásokat használtam, illetve bővítettem az \ref{ch:application-improvement} fejezetben leírtak szerint.

\section{Kutatási cél}
\label{ch:goals}

A cél az, hogy a kutatás során szerzett modell megbízhatóan detektáljon hulladéklerakókat általánosan folyók mentén. Ehhez a \textit{false positive} arányok minél kisebbek kell legyenek, míg a true positive arányok minél nagyobbak. Ugyanakkor nem jelent ugyanakkora problémát egy \textit{false negative}, mint egy \textit{false positive}, mivel a \textit{false positive} eredmények fölöslegesen rossz irányba küldhetik a folyómentő csapatot. 
A kutatólabor 2023-as cikkében bemutatott modell (továbbiakban meglevő modell vagy régi modell) \cite{magyar2023} egyik problémája a nagy \textit{false positive} arányok voltak. Általában a modellnek leginkább az utak, épületek okoznak problémát. Ez annak köszönhető, hogy a hulladék, a törmelék, az épületek, és a föld nagyon hasonló spektrális értékekkel rendelkeznek a használt sávokon. A modell a pusztazámori hulladéklerakóról, illetve a kiskörei víztárolóról szerzett adatokkal volt betanítva. Ezért a következő lépés több adaton betanítani a modellt, nagy figyelmet fektetve arra, hogy az adathalmaz tartalmazzon bőven utakat, épületeket, és más adatokat, amik hasonlítanak a hulladékra. 

\section{A dolgozat felépítése}
A \ref{ch:related_research}. fejezetben bemutatásra kerülnek a hulladékdetektálás témáját feldolgozó kutatások, illetve bemutatom azokat a technikai és számolási eszközöket, melyek lényeges szerepet vállalnak a kutatásomban.
A \ref{ch:training}. fejezetben részletezem a Random Forest modell betanításához előállított adatokat, a modell betanítási paramétereit, illetve megvizsgálok különböző adatfeldolgozási módszereket, ilyen például a főkomponens analízis, képnormalizálás, vízmaszkolás, azzal a céllal, hogy tovább javítsak a modell teljesítményén. A \ref{ch:verification}. fejezetben tárgyalom a modell tesztelésének és validálásának módját, illetve a tesztadatok megválasztásának módját, motivációját. Az \ref{ch:impl}. fejezetben bemutatom a kutatást lényegesen előresegítő szoftveres fejlesztéseket, illetve azt, hogy a kutatás eddigi eredményei miként vannak integrálva a Tiszta-Tisza alkalmazásba. A \ref{ch:sum}. fejezetben összefoglalom a kutatás eredményeit, és ezek alapján tárgyalom a kutatás további lehetséges haladásait. 