\chapter{Bevezetés}
\label{ch:intro}

\todo{szinkronizálni az absztrakttal}A hulladékszennyezés komoly problémát jelent a természet számára \cite{kibria2023PlasticWaste}. Emiatt számos szervezet mozdul abba az irányba, hogy tisztábbá tegye a bolygónkat. Egy ilyen szervezet a PET Kupa, akik folyómenti hulladékgyűjtéssel foglalkoznak elsősorban Magyarországon, de figyelmük kiterjed a szomszédos országokra is. Az egyik nagy kihívás a szemétgyűjtésben a hulladékkal szennyezett területeknek a hatékony megtalálása. Sok erőforrást igényel a hulladéklerakók megtalálása a folyók mentén, hiszen sokszor járművel kell valakinek végig haladnia egy hosszabb területen, azért, hogy felmérje, hogy hol van hulladék. A folyók árterén elhelyezett hulladékok még nagyobb problémát jelentenek, hiszen dagály idejében a hulladékot elmossa a víz és ez a folyó további szakaszaira lesz szétszórva miközben nagy károkat okoz a folyó élővilágának, illetve szennyezi a folyóvizet \cite{nyberg2023, vanEmmerik2023}. Emiatt szükségünk van olyan eszközökre, melyekkel hamar lehet detektálni a szennyezett területeket, hogy ezeket minél hatékonyabban meglehessen tisztítani. Az ELTE Térinformatikai Labor és a PET Kupa együttműködésében olyan eszközöket fejlesztünk, melyek automatikusan képesek lesznek hulladékot detektálni a folyók mentén.

A dolgozatomban bemutatok egy Random Forest modellt \cite{breiman2001}, mely a kutatólaborban már lefejlesztett modellre épül \cite{magyar2023}. A bemutatott modell javít a korábbi megoldás pontosságán, illetve nagyobb megbízhatósággal találja meg a hulladékot a folyókon és a folyók mentén. A modell eredményei integrálásra kerülnek a Tiszta Tisza webalkalmazásba\footnote{A Tiszta-Tisza webalkalmazás hivatalos weboldala a https://tisztatiszaterkep.hu/ címen található, de a dolgozatban említett fejlesztések jelenleg a https://gis.inf.elte.hu/tiszta-tisza/ oldalon érhetőek el.}, ahol több napon keresztül történő detektálás eredménye lesz összesítve és megjelenítve a felhasználók számára. Ezen felül a dolgozatban tárgyalni fogok más kutatást is, mely a hulladékdetektálás problémájával foglalkozik. Ezen kívül kiegészítem a kutatólabor meglevő szoftveres eszközeit annak érdekében, hogy a laborban zajló munka gördülékenyebb legyen.
\todo[inline,color=blue!50]{Hivatkozd meg a https://tisztatiszaterkep.hu/ oldalt, de említsd meg, hogy a prototípus egyelőre a https://gis.inf.elte.hu/tiszta-tisza/ oldalon érhető el. Ez mehet akár csak lábjegyzetbe. Botond: Rendben, lábjegyzetben megemlítettem.}

A kutatás hozzáadott terméke egy olyan adathalmaz, mely alkalmas más hulladékdetektálási modellek betanítására is. Az adathalmaz elsősorban szárazföldi romániai hulladéklerakókról készített PlanetScope műholdfelvételeket tartalmaz, melyek kézzel voltak annotálva. Az adatok georeferálva vannak, így ezeket könnyen meg lehet vizsgálni, illetve ki lehet egészíteni.

Továbbá bemutatom, hogy milyen módszerekkel próbáltam tovább javítani a modell eredményein. Ilyen módszer például a főkomponens analízis, a képnormalizálás, illetve az évszakokra bontás.

\section{A kutatólabor eddigi eredményei}

A ELTE IK Térinformatikai Kutatólaborában már betanításra került egy Random Forest modell, mely folyómenti hulladék detektálására alkalmas. Az én dolgozatomban, a kutatólabor meglevő tudására építve, továbbfejlesztem ezt a modellt, hogy kevesebb false-positive-al találja meg a szeméttel szennyezett területeket. Továbbra implementálásra került egy szerveralkalmazás, mely minden nap a Planet szervereiről letölti a legfrissebb felvételeket a vizsgált területekről, és lefuttatja ezeken a képeken az akkori modellt. Ezen felül készült egy webalkalmazás is, ami erről a szerverről letölti az eredményeket, és megjeleníti ezeket, összehasonlításra. A kutatólabor rendelkezik egy asztali alkalmazással is, mellyel hatékonyan elő lehet állítani tanító adatokat. A kutatásom elősegítéséhez ezeket az alkalmazásokat használtam, illetve bővítettem a \ref{ch:application-improvement} fejezetben leírtak szerint.

\section{Kutatási cél}
\label{ch:goals}

A cél az, hogy a kutatás során szerzett modell megbízhatóan detektáljon hulladéklerakókat általánosan folyók mentén. Ehhez a false positive arányok minél kisebbek kell legyenek, míg a true positive arányok minél nagyobbak. Ugyanakkor nem jelent ugyanakkora problémát egy false negative, mint egy false positive, mivel a false positive eredmények fölöslegesen rossz irányba küldhetik a folyómentő csapatot. 
A kutatólabor 2023-as cikkében bemutatott modell (továbbiakban meglevő modell vagy régi modell) \cite{magyar2023} egyik problémája a nagy false positive arányok voltak. Általában a modellnek leginkább az utak, épületek okoznak problémát. Ez annak köszönhető, hogy a hulladék, a törmelék, az épületek, és a föld nagyon hasonló spektrális értékekkel rendelkeznek a használt sávokon. A modell a pusztazámori hulladéklerakóról, illetve a kiskörei víztárolóról szerzett adatokkal volt betanítva. Ezért egy jó irány több adaton betanítani a modellt, nagy figyelmet fektetve arra, hogy az adathalmaz tartalmazzon bőven utakat, épületeket, és más adatokat, amik hasonlítanak a hulladékra. 

\section{A dolgozat felépítése}
A \ref{ch:related_research}. fejezetben bemutatásra kerülnek a hulladékdetektálás témáját feldolgozó kutatások, illetve bemutatom azokat a technikai és számolási eszközöket, melyek lényeges szerepet vállalnak a kutatásomban.
A \ref{ch:training}. fejezetben részletezem a Random Forest modell betanításához előállított adatokat, a modell betanítási paramétereit, illetve megvizsgálok különböző adatfeldolgozási módszereket, ilyen például a főkomponens analízis, képnormalizálás, vízmaszkolás, azzal a céllal, hogy tovább javítsak a modell teljesítményén. A \ref{ch:verification}. fejezetben tárgyalom a modell tesztelésének és validálásának módját, illetve a tesztadatok megválasztásának módját, motivációját. A \ref{ch:impl}. fejezetben bemutatom a kutatást lényegesen előresegítő szoftveres fejlesztéseket, azt, hogy a kutatás eddigi eredményei miként vannak integrálva a Tiszta-Tisza alkalmazásba. A \ref{ch:sum}. fejezetben összefoglalom a kutatás eredményeit, és ezek alapján tárgyalom a kutatás további lehetséges haladásait. 