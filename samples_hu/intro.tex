\chapter{Bevezetés}
\label{ch:intro}

A hulladékszennyezés komoly problémát jelent a természet számára \cite{kibria2023PlasticWaste}. Emiatt számos szervezet mozdul abba az irányba, hogy tisztábbá tegye a bolygónkat. Egy ilyen szervezet a PET Kupa, akik folyómenti hulladékgyűjtéssel foglalkoznak elsősorban Magyarországon, de figyelmük kiterjed a szomszédos országokra is. Az egyik nagy kihívás a szemétgyűjtésben a szennyezett területeknek a hatékony megtalálása. Sok emberi és pénzügyi erőforrást igényel a hulladéklerakók megtalálása a folyók mentén, hiszen általában járművel kell valakinek végig haladnia egy hosszabb területen, azért, hogy felmérje, hogy hol van hulladék. A folyók árterén elhelyett hulladékok még nagyobb problémát jelentenek, hiszen dagály idejében a hulladékot elmossa a víz és a folyó további szakaszaira lesz szétszórva miközben nagy károkat okoz a folyó élővilágának, illetve szennyezi a folyóvizet. Ennek az  Emiatt hatékonyabb eszközökre van szükségünk, hogy ezt a folyamatot felgyorsítsuk. Ennek fényében a PET Kupa felkereste az egyetemünket azzal a kéréssel, hogy olyan eszközöket fejlesszünk le, melyek automatikusan képesek lesznek hulladékot detektálni a folyók mentén.

A dolgozatomban bemutatok egy Random Forest modell-t\cite{breiman2001}, mely a kutatólaborban már lefejlesztett modellre épül \cite{magyar2023}. A bemutatott modell javít a korábbi modell problémáin, illetve nagyobb megbízhatósággal találja meg a hulladékot a folyókon és a folyók mentén. A modell eredményei integrálásra kerülnek a Tiszta Tisza webalkalmazásba, ahol több napon keresztül történő detektálás eredménye lesz összesítve és megjelenítve a felhasználók számára \todo{Hivatkozni a weboldalra}. Ezen felül tárgyalva lesz több kutatás is, mely a hulladékdetektálás problémájával foglalkozik.

A kutatás hozzáadott terméke egy olyan adathalmaz, mely alkalmas más hulladékdetektálási modellek betanítására is. Az adathalmaz elsősorban szárazföldi Romániai hulladéklerakókról készített PlanetScope műholdfelvételeket tartalmaz, melyek kézzel voltak annotálva. Az annotált adatok georeferálva vannak, így a meglevő adatokat könnyen meg lehet vizsgálni, illetve ki lehet egészíteni. 

Továbbá tárgyalásra kerül több olyan módszer is mellyel tovább próbáltam javítani a modell eredményeit. Ilyen például a Főkomponens analízis, illetve a képnormalizálás, évszakokra bontás.

\subsection{A kutatólabor már meglevő eredményei}

A térinformatikai kutatólaborban már fejlesztésre került egy szerveralkalmazás, mely minden nap a Planet-ről letölti a legfrissebb felvételeket a vizsgált területekről, és lefuttatja ezeken a képeken az akkori modellt. Ezen felül készült egy webalkalmazás is, ami erről a szerverről letölti az eredményeket, és megjeleníti ezeket, összehasonlításra. A kutatólabor rendelkezik egy asztali alkalmazással is, mellyel hatékonyan elő lehet állítani tanítóadatokat. A kutatásom elősegítéséhez ezeket az alkalmazásokat használtam, illetve bővítettem a \ref{ch:application-improvement} fejezetben leírtak szerint.