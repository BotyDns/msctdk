\chapter{Betanítás}
\label{ch:learning}

\subsection{Tanítóadatok}

A betanításhoz 29 romániai hulladéklerakó és közvetlen környezete került a tanítóadatok közé, illetve a Kiskörei víztároló is. A romániai hulladéklerakókat egy helyi weboldalon lehet megtalálni, a hozzájuk tartozó koordinátákkal együtt \cite{wasteromania2019}. Az ott bemutatott 46 hulladéklerakó közül 29 volt alkalmas tanításra: sok hulladéklerakó be lett tömve, vagy föld alatt működik. Minden hulladéklerakóhoz tartozik egy-egy nyári+tavaszi, téli és őszi multispektrális műholdkép, melyeket kézzel annotáltam. A nyári és tavaszi képeket azért vontam egybe, mivel ezek hulladékdetektálás szempontjából hasonló adatokat eredményeztek. A tanítóadatok pixelenként vannak előállítva, így a végső adathalmaz 27 millió tanítóadatból (pixelből) áll. Minden pixelhez hozzá van rendelve a vörös, kék, zöld, közeli infravörös sáv, illetve a "PI", "NDWI", "NDVI", "RNDVI", "SR" indexek. Ezen felül minden pixel címkézve van a \ref{tab:waste-detection-labels} táblázatban leírtak szerint.

\begin{table}[H]
	\centering
	\begin{tabular}{ | p{0.1\textwidth} | p{0.2\textwidth} | p{0.7\textwidth} | }
		\hline
		\textbf{Címke azonosító} & \textbf{Címke neve} & \textbf{Címke magyarázat} \\
		\hline \hline
		\emph{100} & Hulladék & Azon területek, melyeken hulladékot találtunk. \\
		\hline
		\emph{200} & Víz & olyan területek, melyeken kizárólag vizet találtunk, általában folyók. \\
		\hline
		\emph{300} & Legelők/Erdők & Zöld övezetből álló vad területek. Ezek lehetnek fák lombjai vagy füves zónák. \\
		\hline
        \emph{400} & Mezők & Olyan földes területek, melyek meg vannak művelve, illetve ahol mezőgazdasági növények találhatóak, például gabonafélék. \\
		\hline
        \emph{500} & Ismeretlen & Olyan területek, melyek a korábbi kategóriákba nem sorolhatók bele. Ilyenek az épületek, aszfaltozott utak, háztetők, mezei utak. \\
		\hline
	\end{tabular}
	\caption{A tanítóadatok címkéi}
	\label{tab:waste-detection-labels}
\end{table}

\subsection{Tanítási paraméterek}

A nagy adathalmaz miatt a Random Forest modell is nagyon nagy lesz (körülbelül 14GB), ami egy nehezen kezelhető méret, így érdemes módosítani a modell paraméterein, hogy ez kisebb méretű legyen. A legjobb eredményeket azzal értük el, hogy a Random Forest fák méretét 20 mélységűre limitáltuk. Ennek köszönhetően a modellek méretét 2GB-ra tudtuk csökkenteni \todo{táblázat a fák méretéről, a modellek méretéről és a különböző mélységekről}.

Továbbiakban felmerült az a probléma is, hogy a tanítóadatok nagyon aránytalanok voltak: A \ref{fig:unbalanced-data} ábrából látható, hogy kevesebb adattal rendelkeztünk hulladékról, mint az összes többi adatról, így a modell nagyon sok false-negatívot termelt. Ennek korrigálására súlyokat alkalmaztam a tanítóadatokra.

\pgfplotstableread[row sep=\\,col sep=&]{
    label           & value     \\
    Hulladék        & 29513     \\
    Víz             & 926356    \\
    Legelők/Erdők   & 12573615  \\
    Mezők           & 8043948   \\
    Ismeretlen      & 5669416   \\
}\datacounts

\begin{figure}
    \begin{tikzpicture}
        \begin{axis}[
                ybar,
                ymode = log,
                bar width=1cm,
                width=\textwidth,
                height=0.5\textwidth,
                symbolic x coords={Hulladék,Víz,Legelők/Erdők,Mezők,Ismeretlen},
                xtick=data,
            ]
            \addplot table[x=label,y=value]{\datacounts};
        \end{axis}
    \end{tikzpicture}
    \caption{Az adatok közötti aránytalanság, logaritmikus skálázással}
    \label{fig:unbalanced-data}
\end{figure}