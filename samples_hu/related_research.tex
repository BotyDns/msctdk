\chapter{Kapcsolódó kutatások}
\label{ch:related_research}

\subsection{Hulladékdetektálási módszerek}

Számos kutatás készült már a hulladékdetektálás témájában. \cite{sakti2023}-ben Sentinel-2 műholdfelvételeken tanítottak be egy Random Forest modell-t egy indonéziai folyón. A cikkben bevezetik az "Adjusted Plastic Index"-et, mellyel a vegetáció, föld és épületek közötti zajt csökkentik. Ennek az indexnek a kiszámításához a Sentinel-2 műhold piros, közeli infravörös (NIR), illetve rövid hullámhosszú közeli infravörös (SWIR) sávokat használták fel. Validációnak Pleiades műholdképeket és drónfelvételeket klasszifikáltak Mahalanobis távolság gépi tanulási módszerrel. A módszer növényzeten és vízen rendre 88\%, illetve 85\% pontosságot ért el és épületeken, törmeléken és földön rendre 62\%, 53\%, illetve 21\% pontossággal tudta a hulladékot detektálni.

\cite{goncalves2022}-ben Spectral Angle Mapping módszert alkalmaztak multispektrális drónfelvételeken. A célja a kutatásnak az volt, hogy a tengerparton kimosott hulladékot detektálják és klasszifikálják. A módszer alkalmazásához referencia értékeket állítottak elő azzal, hogy elhelyeztek különböző anyagokból álló hulladékot a homokba, és ezekről drónfelvételt készítettek. Ezzel a módszerrel képesek voltak detektálni és klasszifikálni nem csak homok fölött található hulladékot, hanem félig elásott hulladékot is. A 472 kézzel előállított tesztadatból volt a 268 True Positive(57\% összesen), 96 volt a False Positive és 204 volt a False Negative.

\cite{lanorte2017}-ben mezőgazdasági hulladékdetektálásra használtak egy Support Vector Machine modellt, Landsat 8 műholdfelvételeken. A szenzor Kék, Zöld, Piros, NIR, SWIR 1, SWIR 2 és CIRRUS sávját használták a tanítóadatok és tesztadatok előállítására. Ezután véletlenszerűen szétválasztották az adatokat tanítóadatokra és tesztadatokra. A következő osztályokra bontották az adatokat: Hálók, műanyag takarók, föld, növényzet, gyümölcsöskert, olajfás kert, város, fa, fás föld. A modell a tesztadatokat összességében 94\%-os pontossággal tudta klasszifikálni, ahol a legrosszabb arányokat az olajfás kert érte el 77.78\%-os pontossággal.

\subsection{Műholdfelvételek}

\subsection{Random Forest}

