\chapter{Kapcsolódó kutatások}
\label{ch:related_research}

\subsection{Hulladékdetektálási módszerek}

Számos kutatás készült már a hulladékdetektálás témájában. \cite{sakti2023}-ben Sentinel-2 műholdfelvételeken tanítottak be egy Random Forest modell-t egy indonéziai folyón. A cikkben bevezetik az "Adjusted Plastic Index"-et, mellyel a vegetáció, föld és épületek közötti zajt csökkentik. Ennek az indexnek a kiszámításához a Sentinel-2 műhold piros, közeli infravörös (NIR), illetve rövid hullámhosszú közeli infravörös (SWIR) sávokat használták fel. Validációnak Pleiades műholdképeket és drónfelvételeket klasszifikáltak Mahalanobis távolság gépi tanulási módszerrel. A módszer növényzeten és vízen rendre 88\%, illetve 85\% pontosságot ért el és épületeken, törmeléken és földön rendre 62\%, 53\%, illetve 21\% pontossággal tudta a hulladékot detektálni.



\subsection{Műholdfelvételek}

\subsection{Random Forest}

