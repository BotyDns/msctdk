\chapter{Összefoglalás és eredmények}
\label{ch:sum}

\section{A kutatás során elért eredmények}

A dolgozatomban megvizsgáltam több módszert, amivel a korábbi modellben levő kihívásokat korrigáltam a multispektrális Planetscope felvételeken. Előállítottam egy tanító adathalmazt, melynek segítségével meg lehet vizsgálni több gépi tanulási módszert, illetve ki lehet próbálni több képfeldolgozási módszert. Megmutattam, hogy egy nagyobb tanító halmaz segítségével alacsonyabb false-positive arányokkal tudja a Random Forest modell detektálni a hulladékkal szennyezett területeket a teszthalmazban. Kiegészítettem az ELTE IK Térinformatikai laborban használt eszközöket arra, hogy hatékonyabban lehessen előkészíteni és megvizsgálni a különböző hulladékdetektálási módszereket. Felgyorsítottam a meglevő index-számolási algoritmust, annak érdekében, hogy hatékonyabban fel lehessen dolgozni a tanítóadatokat. Megvizsgáltam a főkomponens analízis hatását a Random Forestre, és arra a következtetésre jutottam, hogy a modell láthatóan jobban kezelte PCA segítségével a többdimenziós felvételekben levő zajt, mint a PCA nélküli modell. Továbbá megnéztem, hogy miként teljesít a modell képnormalizálás, illetve vízmaszkolás mellett. Az utóbbi két módszert a laborban kutatják és fejlesztik a kollégáim. Bemutattam a módszer használhatóságát azzal, hogy a Tiszta-Tisza webalkalmazásába integráltam a Random Forest modell eredményeit, egyszerű poligonműveletek segítségével.

\section{A kutatás kihívásai}

A kutatás talán legnagyobb kihívása a tanító adatok megfelelő előállítása és a modellek validációja. A \ref{ch:waste-detection-methods} fejezetben bemutatott hulladékdetektálási módszerekben gyakori módszer volt egy magas felbontású műholdfelvétel használata validációra, vagy nagyon magas felbontású drónfelvételek használata a vizsgált területeken. A modell vizsgálata téli felvételeken is egy kihívás tekintve arra, hogy télen ritkábbak a megfelelő időjárás körülmények a nyári időszakokhoz képest. 

\section{További lépések}

A továbbiakban érdemes megvizsgálni akár személyesen, akár magas felbontású felvételekből a hulladékkal szennyezett területeket pontosabb validáció érdekében. Az adat-normalizálási módszereket is érdemes továbbvizsgálni, tekintve arra, hogy gyakori a zaj a műholdfelvételekben, így ilyen módszerek lényegesen növelhetik a modell megbízhatóságát. További lépésként javasolt akár más gépi tanulási módszereket is kipróbálni, illetve klasszikusabb térinformatikai eszközökkel is megközelíteni a hulladékdetektálás problémáját. Gépi tanulási irány esetén érdemes egy sokkal nagyobb adathalmazt előállítani, pontosabb adatvizsgálat mellett, így a modellek jobban tudják majd általánosítani az adatokat hulladékdetektálás céljából.
