\chapter{Elemzés és tervezés}
\label{ch:spec}

\section{Kutatási cél}
\label{ch:goals}

A cél az, hogy a kutatás során szerzett modell megbízhatóan detektáljon hulladéklerakókat általánosan folyók mentén. Ehhez a false positive arányok minél kisebbek kell legyenek, míg a true positive arányok minél nagyobbak. Ugyanakkor nem jelent ugyanakkora problémát egy false negative, mint egy false positive, mivel a false positive eredmények fölöslegesen rossz irányba küldhetik a folyómentő csapatot. 
A kutatólabor 2023-as cikkjében bemutatott modell (továbbiakban meglevő modell) egyik problémája a nagy false positive arányok voltak. A modell a pusztazámori hulladéklerakóról, illetve a kiskörei víztárolóról szerzett adatokkal volt betanítva. Ezért érdemes első körben egy nagyobb adathalmazzal betanítani a modellt.

\section{Műhold specifikációk}

Az új Random Forest modell a PlanetScope műholdakra van specializálva, azon belül is a legújabb PSB.SD szenzorokra. A modell a Vörös (Red), Kék (Blue), Zöld (Green), és a közeli infravörös (NIR) sávokat használja. A PlanetScope műholdak körülbelül 3 méter/pixel felbontással rendelkeznek \cite{planetsensors2024}.

\section{Használt indexek}

A kutatás során felhasználom a kutatólaborban már számolt indexeket. Pontosabban a Plastic Index (\ref{eq:pi} képlet), Normalized Difference Water Index (\ref{eq:ndwi} képlet), Normalized Difference Vegetation Index (\ref{eq:ndvi} képlet), Reversed Normalized Difference Vegetation Index (\ref{eq:rndvi} képlet), Simple Ratio (\ref{eq:sr} képlet) indexek kerülnek használatra \cite{Themistocleous2020, magyar2023}.

\begin{equation}\label{eq:pi}
    Plastic Index (PI) = \frac{NIR}{NIR + Red}
\end{equation}

\begin{equation}\label{eq:ndwi}
    Normalized Difference Water Index (NDWI) = \frac{Green - NIR}{Green + NIR}
\end{equation}

\begin{equation}\label{eq:ndvi}
    Normalized Difference Vegetation Index (NDVI) = \frac{NIR - Red}{NIR + Red}
\end{equation}

\begin{equation}\label{eq:rndvi}
    Reversed Normalized Difference Vegetation Index (NDVI) = \frac{Red - NIR}{Red + NIR}
\end{equation}

\begin{equation}\label{eq:sr}
    Simple Ratio (SR) = \frac{NIR}{Red}
\end{equation}

