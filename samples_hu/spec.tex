\chapter{Elemzés és tervezés}
\label{ch:spec}

A cél az, hogy a kutatás során szerzett modell megbízhatóan detektáljon hulladéklerakókat. Ehhez a false positive arányok minél kisebbek kell legyenek, míg a true positive arányok minél nagyobbak. Ugyanakkor nem jelent ugyanakkora problémát egy false negative, mint egy false positive, mivel a false positive eredmények fölöslegesen rossz irányba küldhetik a folyómentő csapatot. 
A kutatólabor 2023-as cikkjében bemutatott modell (továbbiakban meglevő modell) egyik problémája a nagy false positive arányok voltak. A modell a pusztazámori hulladéklerakóról, illetve a kiskörei víztárolóról szerzett adatokkal volt betanítva. Ezért érdemes egy nagyobb adathalmazzal betanítani a modellt.

Az új Random Forest modell a PlanetScope műholdakra lesz specializálva, azon belül is a legújabb PSB.SD szenzorokra. A modell számára elérhető lesz a Vörös, Kék, Zöld, és a közeli infravörös (NIR) sáv. A PlanetScope műholdak körülbelül 3 méter/pixel felbontással rendelkeznek \cite{planetsensors2024,planetresolution2024}.